\documentclass[12pt]{report}

\usepackage[english]{babel}
\usepackage[utf8x]{inputenc}
\usepackage{amsmath}
\usepackage{graphicx}
\usepackage{multirow}
\usepackage[hypcap]{caption}
\usepackage{setspace} 

\title{Lab 4: Hardness Test, Phase Diagrams, and Heat Treatment}
\author{Zachary Tschirhart \\
	\small \\
        \small EID: zst75 \\
	\small Department of Aerospace Engineering and Engineering Mechanics \\
	\small \textbf{ASE 324L (Mon 2:00-5:00)} \\
	\small Unique: 13740}

\date{February 17, 2014}


\begin{document}
\maketitle

\setcounter{secnumdepth}{0}

\section{Results and Discussion}
\doublespacing

\subsection{Question 1a}

This is a solid two phase equilibrium, with both \(\alpha\) and \(Al_5\) \(Mg_8\).

\subsection{Question 1b}

Since the \(\alpha\) line at 200 degrees crosses at around the 2.5\% weight mark and at the 35\% for the \(\beta\) line. The percentage of solution in the (Al) phase is 2.5\% and (Mg) is 35\%.

\subsection{Question 1c}

From the values found above, we can use the Lever rule equation to find out the percentage of weight:

\begin{equation}
\begin{split}
  \overline{m}_{\alpha} = \frac{m_{\alpha}}{m_0} = \frac{C_0 - C_L}{C_{\alpha} - C_L} 
  = \frac{5.5 - 35}{2.5 - 35} = 90.76\% 
\end{split}
\end{equation}
\\

Meaning, that 90.76 Wt\% is in the \(\alpha\) phase, while 9.24 Wt\% is in the \(\beta\) phase.

\subsection{Question 1d}

When you slowly cool down the solid solution from 450 degrees to 20 degrees Celsius, the solution doesn't form smaller uniform grains as a quickly quenched solution. This solution is in equilibrium. 

\subsection{Question 1e}

When this solution is quickly quenched and misses the nose in the TTT diagram, the microstructue is not in equilibrium and is super saturated, causing smaller grains.

\subsection{Question 2a}




\subsection{Question 2b}


\subsection{Question 2c}


\subsection{Question 3}


\subsection{Question 4}


\subsection{Question 5}


\end{document}
